\documentclass[line, margin]{res}
\usepackage[inline]{enumitem}


\topmargin=-0.5in
\oddsidemargin -.5in
\evensidemargin -.5in
\textwidth=6.0in
\itemsep=0in
\parsep=0in


 
 % For some reason this gives me the format I want when included.  Will debug later...

$ $
 

\usepackage{fontspec}
% \setmainfont[Ligatures=TeX]{HelveticalNeue.ttc}
\begin{document}
\name{Evan Putnam \hspace{24.3em} \small github.com/EvanMPutnam }
\address{(484) 554-8487\\ emp9173@rit.edu}

\begin{resume}

\section{OBJECTIVE}
3rd year computer science major looking to apply learned theory and obtain practical experience through internship employment. Available summer 2019.

\section{EDUCATION} 
 Rochester Institute of Technology \\
 Bachelor of Science in Computer Science, May 2021\\
 \textbf{GPA:} 3.43
 
\section{SKILLS}
Proficient in C, Java, Python, SQL, Git, JetBrains IDE’s, Mac OS, Windows, and Linux. \\
Experience in Swift, C\#, C++, Lua, Javascript, HTML, CSS, Bootstrap, MongoDB, Neo4J, Visual Studio, and Unity3D.
 
\section{EXPERIENCE} 
\textbf{Guardian Life IT and Shared Services Intern}, \textit{Bethlehem PA} \hfill Jun 2018-Current \\
Worked on development of an enterprise Java web application that used Spring MVC, Bootstrap, and JQuery.  In addition created
data analytics for software incident reports, wrote software to visualize that data, and automated database tasks.\\ [10pt]
\textbf{RIT Computer Science Student Lab Instructor}, \textit{Rochester NY} \hfill Jan-May 2018 \\
Helped students understand core concepts of CSII. Responsible for writing scripts
to grade labs, conducting office hours for students with questions, and helping students
with in-lab exercises.

\section{PROJECTS}
\textbf{Augmented Reality Star Viewer}, \textit{WWDC SCHOLARSHIP 2018}\\
Designed and Implemented an Augmented Reality application that won a 2018 WWDC Scholarship from Apple.  The application was created in Swift and used data from the open source HYG-Star database to create an AR walkthrough of the known universe.
\\ [10pt]
\textbf{Meade LX-200 Automation Software}, \textit{RIT SPEX}\\
Designed and wrote software in Python for automation of Meade LX-200 telescope. Server device connects and sends commands, that are dictated by the client, to telescope via serial port over wireless network. Features movement with joystick and tracking of low earth orbit satellites with two line elements.
\\ [10pt]
\textbf{Kladoi Language Compiler}, \textit{SCHOOL PROJECT}\\
As a final project for Programming Language Concepts a compiler written in Python was designed and implemented to convert a BASIC-like language to an assembly-like language.  It involved building a custom lexer and recursive descent parser, as well as creating abstract syntax trees for evaluation with the parser.
\\ [10pt]
\textbf{Sudoku Solver}, \textit{PERSONAL PROJECT}\\
Designed and wrote software in Python for a console application that solves any 9x9 Sudoku puzzle with recursive backtracking.  It was then converted into a web application using Javascript, HTML, and CSS.


\section{ACTIVTIES}
\textbf{RIT Space Exploration (RIT SPEX)}\\
Treasurer
      \begin{enumerate}
        \item[] Responsible for managing funds of RIT SPEX.
      \end{enumerate}
      
Communication Sub-System Team Leader
      \begin{enumerate}
        \item[] Responsible for leading a team in the design of a communications sub-system for a small cube satellite.
        \item[] Gathered weather images from NOAA satellites in which antennas were designed and implemented. Wrote software for automation of process.
      \end{enumerate}
      

\textdollar 50 Satellite Team Leader
      \begin{enumerate}
        \item[] Led a team in the design and preliminary construction of a small satellite based on design from the \textdollar 50SAT - Eagle2 project.
      \end{enumerate}

\section{COURSEWORK}


      \begin{enumerate*}[series=MyList, before=\hspace{-0.6ex}, label=\textbullet]
        \item Introduction to Software Engineering
        \item Mechanics of Programming
        \item Intro to Computer Science Theory
        \item Principles of Data Management
        \item Programming Language Concepts
        \item Computer Science II
        \item Discrete Math for Computing
        \item Probability and Statistics I
        \item Linear Algebra
        \item Calculus II
        \item Business Communications
      \end{enumerate*}


 
\end{resume}
\end{document}
