\documentclass[line, margin]{res}
\usepackage[inline]{enumitem}


\topmargin=-0.5in
\oddsidemargin -.5in
\evensidemargin -.5in
\textwidth=6.0in
\itemsep=0in
\parsep=0in
 
 % For some reason this gives me the format I want when included.  Will debug later...
\bottommargin $ $
 
\begin{document}
\name{Evan Putnam \hspace{16.6em} \small evanputnam.com, github.com/EvanMPutnam }
\address{(484) 554-8487\\ emp9173@rit.edu}





 
 
 
\begin{resume}
\section{EDUCATION} 
 Rochester Institute of Technology \\
 Bachelor of Science in Computer Science, May 2021\\
 \textbf{GPA:} 3.4
 
\section{TECH}
Proficient in Python, Java, C, SQL, JetBrains IDE’s, Microsoft Office Suite, Windows, Mac OSX, and Linux. \\
Experience in C\#, C++, Lua, Javascript, HTML, CSS, MongoDB, Neo4J, Git, Visual Studio, and Unity3D.
 
\section{EXPERIENCE} 
\textit{\textbf{RIT Computer Science Student Lab Instructor}, Rochester NY} \hfill Jan-May 2018 \\
Helped students understand core concepts of CSII. Responsible for writing scripts
to grade labs, conducting office hours for students with questions, and helping students
with in-lab exercises.\\ [10pt]
\textit{\textbf{Psychological Counseling Services}, Bethlehem PA} \hfill May 2016-October 2017 \\
Performed duties of a technical assistant in developing social media automation
software, implementing security features, maintaining site content, and
search engine optimization.
 
\section{PROJECTS}
\textit{\textbf{Meade LX-200 Automation Software}, RIT SPEX}\\
Designed and wrote software in Python for automation of Meade LX-200 telescope. Server device connects and sends commands, that are dictated by the client, to telescope via serial port over wireless network. Features movement with joystick and tracking of low earth orbit satellites with two line elements.
\\ [10pt]
\textit{\textbf{Star Viewer}, RIT SPEX}\\
Designed and wrote software in C\# and Unity to create a star field from data contained in the HYG Database. It parses the data from a csv to get the cartesian coordinates for general location and b-v value for color. Then it is plotted on a 3-D coordinate system and viewable with mouse and keyboard. The star field can also be flown around with a Joystick.
\\ [10pt]
\textit{\textbf{C language FRED Interpreter}, SCHOOL PROJECT}\\
Designed an interpreter for a BASIC like programming language. Work included tokenizing language
keywords, evaluating mathematical expressions in infix form, creating a custom stack, and creating a custom heap. Language features include variables, loops, and conditionals.
\\ [10pt]
\textit{\textbf{Sudoku Solver}, PERSONAL PROJECT}\\
Designed and wrote software in Python for a console application that solves any 9x9 Sudoku puzzle with recursive backtracking.


\section{ACTIVTIES}
\textit{\textbf{RIT Space Exploration(RIT SPEX)}}\\
Treasurer
      \begin{enumerate}[series=MyList, label=\textbullet]
        \item Responsible for managing funds of RIT SPEX.
        \item Conducted research on various crowdfunding techniques.
      \end{enumerate}
      
Communication Sub-System Team Leader
      \begin{enumerate}[series=MyList, label=\textbullet]
        \item Responsible for leading a team in the design of a communications sub-system for a small cube satellite.
        \item Gathered weather images from NOAA satellites in which antennas were designed and implemented. Wrote software for automation of process.
      \end{enumerate}
      

\textdollar 50 Satellite Team Leader
      \begin{enumerate}[series=MyList, label=\textbullet]
        \item Leading a team that is currently designing and constructing a version of a small satellite based on design from the \textdollar 50SAT - Eagle2 project.
      \end{enumerate}

\section{COURSEWORK}


      \begin{enumerate*}[series=MyList, before=\hspace{-0.6ex}, label=\textbullet]
        \item Introduction to Software Engineering
        \item Mechanics of Programming
        \item Intro to Computer Science Theory
        \item Principles of Data Management
        \item Programming Language Concepts
        \item Computer Science II
        \item Discrete Math for Computing
        \item Probability and Statistics I
        \item Linear Algebra
        \item Calculus II
        \item Business Communications
      \end{enumerate*}


 
\end{resume}
\end{document}
